\documentclass[12pt]{article}
%\documentclass[a4paper,14pt]{extarticle}

% Space between paragraphs
\setlength{\parskip}{12pt plus1pt}

%\setlength{\oddsidemargin}{0in}
%\setlength{\evensidemargin}{0in}
%%\setlength{\textwidth}{6.5in}
%\setlength{\topmargin}{-.3in}
%\setlength{\textheight}{9in}
%\pagestyle{empty}

%\usepackage[left=1in,
%			right=1in,
%			top=0.8in,
%			bottom=0.8in,
%			headsep=.15in,
%			footskip=.25in]{geometry}
\usepackage[margin=1in]{geometry}
%\usepackage[margin=1in, headsep=.2in, footskip=.2in]{geometry}

% Insert image
\usepackage{graphicx}

% colors
\usepackage[usenames, dvipsnames]{xcolor}

% For urls
\usepackage[colorlinks=true]{hyperref}

\hypersetup{
	citecolor=magenta,
	linkcolor=Mahogany,
	urlcolor=Blue
}
\usepackage{fancyhdr}
\pagestyle{fancy}
\fancyhf{}
%\rhead{Siddharth Saha}
%\lhead{Stanford MS in Mechanical Engineering}
%\lhead{Siddharth Saha}
%\rhead{CMU MSR}
\rhead{Siddharth Saha}
\lhead{Personal Statement for Robotics M.S., U-M $|$ ID: --------}
%\lhead{University of California, Los Angeles}
\rfoot{Page \thepage}

\begin{document}
	
%	\begin{center}
%		
%		{\Large Personal Statement} \\[.08in]
%		{\large of Siddharth Saha (MS in Mechanical applicant for Fall 2022)}
%		%{\Large Statement of Purpose} \\[.08in]
%		%{\large of Siddharth Saha (MSR applicant for Fall—2022)}
%	\end{center}
%\vspace{-0.08in}

\vspace*{-.3in}


India is a land of contradictions. We celebrate our economic progress, yet squalor and backwardness stand out. Poverty pushes people towards degrading and dangerous jobs. I have directly observed men entering street sewers for cleaning and unclogging. Such sights disturbed me, but I didn’t have an answer then. During college, I realized that robotics could tackle hazardous jobs. As an example, I founded a quadruped team motivated to solve the Robocup Rescue League. Its challenge is to build a search and rescue robot that could serve as a first responder, a job that endangers human life. After college, I had the option to continue in a cushy finance job, ignoring my aptitude and college efforts in robotics. Instead, I want to play a part in solving real-world problems by advancing robotics capabilities. These occupational hazards are universal problems exacerbated by poor infrastructure and overpopulation in developing countries. I wish to bring this perspective into my graduate class at Michigan. I believe that the first-responder robots project led by Prof. Grizzle and Prof. Ghaffari aligns very closely with my aspirations. For these reasons, I am keen to pursue a graduate degree at the U-M Robotics Institute.

My upbringing in a developing nation has also exposed me to the lack of access to quality education. While this issue is highly apparent in developing countries, inequitable access to quality education is a universal problem. The disparity in skill-based educational access is starker in the field of robotics. Despite studying at IIT Bombay (best-ranked in India), I experienced that robotics education is very unstructured at the college level — scrambling for relevant courses across multiple departments while avoiding time slot conflicts and gaining experience through self-projects. My Google Summer of Code (GSoC) ’21 was at JdeRobot, which addresses the high learning curve in robotics, making it more accessible to beginners. After GSoC, I have continued as a maintainer of JdeRobot’s framework. In the future, I wish to use my expertise to develop educational technology in pursuit of robotics excellence. Increasing accessibility to robotics learning will remove one of the barriers for underrepresented students and help draw out talented students in the field. As the field of robotics is made more accessible, the resulting technological progress will further contribute to reducing hazard levels in manual jobs.





\setlength{\parskip}{0pt plus1pt}
%%% V1 %%%%%
%\vspace{15pt}
%Sincerely yours,\par
%\vspace{15pt}
%\rule{9em}{0.5pt}\par
%Mumbai, November 27, 2021

%%%% V2 %%%%%
%\vspace{10pt}
%\noindent Sincerely yours, \par
%\vspace{15pt}
%\noindent\rule{12em}{0.5pt}\par
%\noindent Mumbai, November 27, 2021

\vspace{15pt}
\noindent Sincerely yours,\par
%\vspace{15pt}
%\vspace{5pt}
%\noindent\rule{12em}{0.5pt}\par
%\includegraphics[width=0.07\linewidth]{sign.jpeg}\par
\noindent Siddharth Saha \par
%\vspace{5pt}
\noindent Mumbai, January 14, 2022






\end{document}
